\documentclass[dvipdfmx,cjk]{beamer}

%=====
% 設定(要編集)
%=====
\usetheme{Boadilla}% スライドのテーマ(様式)の指定
\usecolortheme{whale}% カラーテーマの指定
\newcommand{\happyou}{テンプレ説明書}
\newcommand{\JTITLE}{スライドテンプレ説明書}%
\newcommand{\ETITLE}{Instructions for slide template}%
\newcommand{\LastName}{内野}%
\newcommand{\FirstName}{佑基}%
\newcommand{\MyNumber}{}%
\newcommand{\DATE}{\today}%
\newcommand{\Supervisora}{}%
\newcommand{\Supervisorb}{}%
\newcommand{\Position}{}% 指導教員の職位
\newcommand{\Labo}{}%
\newcommand{\Daigaku}{}%
\newcommand{\Gakubu}{}%
\newcommand{\Gakka}{}%

%=====
% パッケージの読み込み
%=====
\usepackage{graphicx}
\usepackage{color}
\usepackage{bm}
\usepackage{url}
\usepackage{amsthm,amsmath,amssymb}
\usepackage{ascmac}
\usepackage{cases}
\usepackage{mathrsfs}
\usepackage{tikz}
\usepackage{algpseudocode}
\usepackage{tcolorbox}
\usepackage{varwidth}
\usepackage{mathtools}
\usepackage{multirow}
\usepackage{multicol}

%=====
% その他の設定(編集不要)
%=====
% PDFアウトラインの文字化け修正
\usepackage{atbegshi}
\usepackage{pxjahyper}
\ifnum 42146=\euc"A4A2 \AtBeginShipoutFirst{\special{pdf:tounicode EUC-UCS2}}\else
\AtBeginShipoutFirst{\special{pdf:tounicode 90ms-RKSJ-UCS2}}\fi

% 右下のアイコンを消す
\setbeamertemplate{navigation symbols}{} 

% フレームタイトルの設定(center:中央寄せ)
\setbeamertemplate{frametitle}[default][center]

% スライド下段のフォントの設定
\setbeamerfont{footline}{size=\scriptsize,series=\bfseries}

% 全体のフォントの設定
\renewcommand{\kanjifamilydefault}{\gtdefault}
\renewcommand{\familydefault}{\sfdefault}
\usefonttheme[onlymath]{serif}

% tikzで使うやつ
\usetikzlibrary{shapes,snakes}
\usetikzlibrary{positioning, decorations.pathreplacing, arrows.meta}
\usetikzlibrary{shadings,shadows,shapes.arrows}

% 矢印コマンドの作成
\newcommand*{\Rarrow}[2]{%
	\tikz[
	baseline=(A.base),
	font=\footnotesize\sffamily
	]
	\node[
	single arrow,
	single arrow head extend=4pt,
	draw=#1,
	inner sep=2pt,
	top color=#1,
	bottom color=#1,
	] (A) {#2};%
}
\newcommand*{\Darrow}[2]{%
	\tikz[
	baseline=(A.base),
	font=\footnotesize\sffamily
	]
	\node[
	single arrow,
	single arrow head extend=4pt,
	single arrow tip angle=150,
	shape border rotate=270,
	inner sep=2pt,
	top color=#1,
	bottom color=#1,
	] (A) {#2};%
}
\newcommand*{\DArrow}[1]{%
	\begin{center}\hspace{-\csname @totalleftmargin\endcsname}
		\noindent\tikz[
		baseline=(A.base),
		font=\footnotesize\sffamily
		]
		\node[
		single arrow,
		single arrow head extend=5pt,
		single arrow tip angle=120,
		shape border rotate=270,
		inner sep=2pt,
		top color=#1,
		bottom color=#1,
		] (A) {\large\color{#1}aaa};%
	\end{center}
}

% 文字の色を変更するコマンドの作成
\newcommand{\blue}[1]{{\color{blue!95!black}{#1}}}
\newcommand{\red}[1]{{\color{red!90!black}{#1}}}
\newcommand{\green}[1]{{\color[RGB]{10,79,61}{#1}}}

% 定理環境
\setbeamertemplate{theorems}[numbered]
\newtheorem{thm}{定理}
\newtheorem{defi}[thm]{定義}
\newtheorem{lem}[thm]{補題}
\newtheorem{prop}[thm]{命題}
\newtheorem{coro}[thm]{系}
\newtheorem{algo}[thm]{アルゴリズム}
\renewcommand\proofname{\textbf{証明}}
\newtheorem{exam}{例}

% 文章中の数式($~~$)を1行幅に縮小しないための設定(分数など)
\everymath{\displaystyle}

% 図表のキャプションの変更
\renewcommand{\figurename}{図}
\renewcommand{\tablename}{表}
\setbeamertemplate{caption}[numbered]

% algorithmicの設定
\makeatletter
\renewcommand{\ALG@beginalgorithmic}{\tt}
\makeatother
\algrenewcommand\alglinenumber[1]{{\scriptsize\texttt{#1:}}}

% 参考文献を下部に表示するコマンドの作成
\newcommand{\BOOK}[2]{%
	\renewcommand\thefootnote{}%
	\renewcommand{\arraystretch}{0.5}
	\onslide\footnotetext{
		\vspace{-1.2em}
		{\tiny\begin{tabular}{p{.025\hsize}p{.88\hsize}}
			\hspace{-.5em}
			#1$\!$&$\!$#2
		\end{tabular}}
	}%
	\renewcommand{\arraystretch}{1.2}
	\renewcommand\thefootnote{*\arabic{footnote}}%
}
\renewcommand{\footnoterule}{\vspace{2mm}\hrulefill\vspace{1mm}}

% 総ページ数の制御
% (\backupbegin~\backupend内のページはカウントしない)
\newcommand{\backupbegin}{
	\newcounter{framenumberappendix}
	\setcounter{framenumberappendix}{\value{framenumber}}
}
\newcommand{\backupend}{
	\addtocounter{framenumberappendix}{-\value{framenumber}}
	\addtocounter{framenumber}{\value{framenumberappendix}}
}

% オリジナルのBOX
\newenvironment{mybox}[2][blue]{%
\ifstrempty{#2}{%
\begin{tcolorbox}[colback=#1!3,colframe=#1!50!black,left skip = 0cm-\csname @totalleftmargin\endcsname,right skip = 0cm,boxrule=0.4mm]}{%
\begin{tcolorbox}[colback=#1!3,colframe=#1!50!black,title={\Large #2},left skip = 0cm-\csname @totalleftmargin\endcsname,right skip = 0cm,boxrule=0.4mm]}}{%
\end{tcolorbox}}

% \left,\rightの余白の設定
\let\originalleft\left
\let\originalright\right
\renewcommand{\left}{\mathopen{}\mathclose\bgroup\originalleft}
\renewcommand{\right}{\aftergroup\egroup\originalright}

% タイトルの設定
\title[\happyou]{\JTITLE}
\subtitle{\textbf{\ETITLE}}
\author[\LastName\ \FirstName]{\LastName\ \FirstName}
\date{\DATE}
%=====
% その他の設定ここまで
%=====
\begin{document}


% タイトルページ
\begin{frame}[plain]
	\titlepage %表紙
\end{frame}


% 目次
\begin{frame}{目次}
	\tableofcontents
\end{frame}


\section{スライドのデザイン(テーマ)について}
\begin{frame}{スライドテーマ}
\begin{itemize}
\item スライドのデザインや色は任意に変更可能
\item 変更する場合は以下を参照してください
\item スライドのテーマ(様式)の指定

\url{https://deic.uab.cat/~iblanes/beamer_gallery/index_by_theme.html}

\item カラーテーマの指定

\url{https://deic.uab.cat/~iblanes/beamer_gallery/index_by_color.html}

\item 自己流にカスタマイズもできます
\end{itemize}
\end{frame}


\section{フレームについて}
\begin{frame}[containsverbatim]{フレームについて}
\begin{itemize}
\item 以下のようにしてスライドの各ページを作成する
\end{itemize}

\begin{mybox}{}
\begin{verbatim}
\begin{frame}{フレームタイトル}
... 1ページ目内容 ...
\end{frame}

\begin{frame}{フレームタイトル}
... 2ページ目内容 ...
\end{frame}
...
\end{verbatim}
\end{mybox}

\begin{itemize}
\item 読みやすさのため,\red{内容は箇条書きにする}
\item \verb+\verb||+などの文字列をそのまま出力するコマンドを含む場合は

 \verb|\begin{frame}[containsverbatim]{フレームタイトル}|

としないとエラーになる
\item スライドの書き方のサンプルがほしい場合は内野まで
\end{itemize}
\end{frame}


\section{目次について}
\begin{frame}[containsverbatim]{目次について}
\begin{itemize}
\item \verb|\section|, \verb|\subsection|で記述したものは目次に反映される
\item 以下の様に書くと「1. セクション1」「2. セクション2」が目次に追加される
\item 目次は「\verb|\tableofcontents|」コマンドで表示できる
\item \verb|\section*{}|と書くと目次ページには表示されなくなる(PDFの目次には反映される)
\end{itemize}

\begin{mybox}{}\footnotesize
\begin{verbatim}
\section{セクション1}
\begin{frame}{フレームタイトル}
... 1ページ目内容 ...
\end{frame}

\section{セクション2}
\begin{frame}{フレームタイトル}
... 3ページ目内容 ...
\end{frame}
\end{verbatim}
\end{mybox}
\end{frame}


\section{アニメーションについて}
\begin{frame}[containsverbatim]{アニメーション}
\begin{itemize}
\item \verb|\pause|を使用するとスライドにアニメーションを追加できる
\item 例えば
\begin{mybox}{}
\begin{verbatim}
... 内容1 ... \pause
... 内容2 ... \pause
... 内容3 ... 
\end{verbatim}
\end{mybox}
のように書くと以下のようになる
\end{itemize}

\noindent
\begin{minipage}[b]{.32\hsize}
\begin{mybox}{1ページ目}
\begin{verbatim}
... 内容1 ... 


\end{verbatim}
\end{mybox}
\end{minipage}
\begin{minipage}[b]{.32\hsize}
\begin{mybox}{2ページ目}
\begin{verbatim}
... 内容1 ... 
... 内容2 ... 

\end{verbatim}
\end{mybox}
\end{minipage}
\begin{minipage}[b]{.32\hsize}
\begin{mybox}{3ページ目}
\begin{verbatim}
... 内容1 ...
... 内容2 ...
... 内容3 ... 
\end{verbatim}
\end{mybox}
\end{minipage}
\end{frame}


\begin{frame}[containsverbatim]{アニメーション}
\begin{itemize}
\item \verb|itemize|にはアニメーションが備わっているので\verb|\pause|を使用する必要はない
\end{itemize}
\begin{mybox}{}
\begin{verbatim}
\begin{itemize}
\item<1-3>      1ページ目から3ページ目までに表示
\item<1->       1ページ目から最後のページまでに表示
\item<3>        3ページ目のみに表示
\item<-5>       5ページ目までに表示
\item<1-2,4,6-> 1,2,4ページ目と6ページ目以降に表示
\end{itemize}
\end{verbatim}
\end{mybox}
\end{frame}


\section{箇条書き補足}
\begin{frame}[containsverbatim]{箇条書き補足}
\begin{itemize}
\item \verb|\item[文字]|とすると箇条書きの黒丸を文字に変更できる
\item ただし,文字は右揃えなので長い文字は左がはみ出る
\end{itemize}
\begin{mybox}{}
\begin{verbatim}
\begin{itemize}
\item                1文目
\item[$\Rightarrow$] 2文目
\item[長い文字]       3文目
\end{itemize}
\end{verbatim}
\end{mybox}
\begin{itemize}
\item 1文目
\item[$\Rightarrow$] 2文目
\item[長い文字] 3文目
\end{itemize}
\end{frame}


\section{箇条書き内の数式}
\begin{frame}[containsverbatim]{箇条書き内の数式}
\begin{itemize}
\item 箇条書き内に別行立ての数式を書くと右にずれてしまう
\item そこで,\verb|\hspace{-2em}|で左に少しずらす必要がある
\item 箇条書きが2重の場合は\verb|\hspace{-4em}|とする
\end{itemize}
\begin{mybox}{}\footnotesize
\begin{verbatim}
\begin{itemize}
   \item なんかの文章
   \[\hspace{-2em}             % 左に2emずらす
      数式
   \]
\end{itemize}
\end{verbatim}
\end{mybox}

\begin{itemize}\footnotesize
\item 1重の箇条書き内の場合:
\[
Ax = b\tag{\texttt{hspace}無し}
\]
\[\hspace{-2em}
Ax = b\tag{\texttt{hspace\{-2em\}}}
\]
\begin{itemize}\footnotesize
\item 2重の箇条書き内の場合:
\[
Ax = b\tag{\texttt{hspace}無し}
\]
\[\hspace{-4em}
Ax = b\tag{\texttt{hspace\{-4em\}}}
\]
\end{itemize}
\end{itemize}
\end{frame}


\section{定理型環境について}
\begin{frame}[containsverbatim]{定理型環境}

\begin{itemize}
\item \verb|thm|定理,\verb|defi|定義,\verb|lem|補題,\verb|prop|命題,\verb|coro|系,\verb|algo|アルゴリズム,\verb|proof|証明,\verb|exam|例の環境を用意している
\item スライドテーマによりデザインが変わる
\end{itemize}

\vspace{-2mm}
\begin{mybox}{}
\begin{verbatim}
\begin{thm}[定理タイトル]
	これは定理環境(タイトルあり)です.
\end{thm}
\begin{thm}
	これは定理環境(タイトル省略)です.
\end{thm}
\end{verbatim}
\end{mybox}
\vspace{-3mm}
\begin{thm}[定理タイトル]
	これは定理環境(タイトルあり)です.
\end{thm}

\begin{thm}
		これは定理環境(タイトル省略)です.
\end{thm}
\end{frame}


\section{オリジナルコマンドについて}
\begin{frame}[containsverbatim]{オリジナルコマンド(文字色,枠)}
\begin{itemize}
\item \verb|\red{文字}|,\verb|\blue{文字}|,\verb|\green{文字}|で文字の色を赤,青,緑に変更可能(普通のred, blue, greenだと色が薄くて見えずらいので少し黒くしている)
\begin{itemize}
\item \verb|\red{あああ}| $\to$ \red{あああ}
\item \verb|\blue{あああ}| $\to$ \blue{あああ}
\item \verb|\green{あああ}| $\to$ \green{あああ}
\end{itemize}
\item \verb|\begin{mybox}[色]{タイトル}~\end{mybox}|で色付きの枠が使用可能
\end{itemize}

\begin{mybox}[green]{タイトル}
	\verb|\begin{mybox}[green]{タイトル}|
\end{mybox}

\begin{mybox}[blue]{}
	\verb|\begin{mybox}[blue]{}% タイトルを省略するとこうなる|
\end{mybox}
\end{frame}


\begin{frame}[containsverbatim]{オリジナルコマンド(参考文献)}
\begin{itemize}
	\item \verb|\BOOK{参考文献の\cite}{参考文献の情報}|でフレーム下部に参考文献を表示できる(\red{文献の初出時に必ず書くこと}).
	\item 同ページに複数ある場合は各文献で\verb|\BOOK|を使用する\\
	 \verb|\BOOK{参考文献1の\cite}{参考文献1の情報}|\\
	 \verb|\BOOK{参考文献2の\cite}{参考文献2の情報}|
	\item \red{文献情報は必ず参考文献のページにも記述する}.
\end{itemize}
	
\vspace{-2mm}
\begin{mybox}{}\footnotesize
\begin{verbatim}
\begin{frame}
   ...IEEE 754規格~\cite{IEEE}に基づく.

   \BOOK{\cite{IEEE}}{%
      ANSI/IEEE Std 754-2008, 
      \textit{IEEE Standard for Floating Point Arithmetic}, 
      IEEE, 2008.}
\end{frame}
\end{verbatim}
\end{mybox}

...IEEE 754規格~\cite{IEEE}に基づく.

\BOOK{\cite{IEEE}}{%
	ANSI/IEEE Std 754-2008, 
	\textit{IEEE Standard for Floating Point Arithmetic}, 
	IEEE, 2008.}
\end{frame}


\begin{frame}[containsverbatim]{オリジナルコマンド(矢印)}
	
	\begin{itemize}
		\item 色付きの矢印を複数用意している
		\begin{itemize}
		\item \verb|\Rarrow{色}{文章}| : 文字を含む右向きの矢印
		\item \verb|\Darrow{色}{文章}| : 文字を含む下向きの矢印
		\item \verb|\DArrow{色}      | : 中央寄せされた下向きの矢印
		\end{itemize}
	\end{itemize}
	
	\begin{mybox}{}\footnotesize
	\begin{verbatim}
		\Darrow{cyan!70!blue}{そこで},    % 右向き
		\Rarrow{red!50!yellow}{ゆえに},   % 下向き
		\Rarrow{red}{\color{red}aaa},     % 文字を同じ色にした場合
		\DArrow{red}                       % 下向き中央寄せ
	\end{verbatim}
	\end{mybox}
	
	\Darrow{cyan!70!blue}{そこで},
	\Rarrow{red!50!yellow}{ゆえに},
	\Rarrow{red}{\color{red}aaa}
	\DArrow{red}
\end{frame}


\section{参考文献について}
\begin{frame}[containsverbatim]{参考文献}
\begin{itemize}
\item \verb|\begin{frame}[allowframebreaks]{参考文献}|のように書くと,ページからあふれたときに自動で改ページされる

\item このとき,フレームタイトルは参考文献I, 参考文献IIのようになる

\item 前述の通り,参考文献は引用したページにも書く(\verb|\BOOK|コマンド)

\item 参考文献の書き方はゼミ資料と同じ

\item Beamerのデフォルトの設定では番号が表示されないので,\\
 \verb|\bibliographystyle{apalike}|\\
 \verb|\beamertemplatetextbibitems|\\
を記述して番号が表示されるようにする.(テンプレでは変更済み)
\end{itemize}
	
\begin{thebibliography}{99}\footnotesize
\bibliographystyle{apalike}
\beamertemplatetextbibitems

\bibitem{IEEE}
ANSI/IEEE Std 754-2008, 
\textit{IEEE Standard for Floating Point Arithmetic}, 
IEEE, 2008.

\bibitem{rump2008}
S.M. Rump, T. Ogita, S. Oishi, 
Accurate Floating-Point Summation Part II: Sign, $K$-Fold Faithful and Rounding to Nearest, 
\textit{SIAM J. Sci. Comput.}, \textbf{31}:2 (2008), 1269--1302.
\end{thebibliography}

\end{frame}


\section{tikzpictureの応用}
\begin{frame}[containsverbatim]{tikzpictureの応用1}
\begin{itemize}
\item \verb|\begin{tikzpicture}[overlay,remember picture]|を使用することで,スライドの任意の位置に文字や図を上から重ねることができる
\item 以下は,記号の説明をフレームの右上(north east)に表示する例
\begin{itemize}
\item 表示する内容(ここから)~表示する内容(ここまで)を編集すれば使用可能
\end{itemize}
\end{itemize}


\begin{mybox}{}\scriptsize
\begin{verbatim}
\begin{tikzpicture}[overlay,remember picture]
\draw (current page.north east)               % このページのnorth east
  node[fill=white,draw=black,below left]{%    % ノードを書く
    \begin{varwidth}{10cm}\footnotesize       % 可変幅のminipage(最大10cm)
      \begin{tabular}{rl}                     % 表示する内容(ここから)
        $\mathbf{u}$         : & 単位相対丸め\\
        $\mathrm{fl}(\cdot)$ : & 数値計算結果\\
      \end{tabular}                           % 表示する内容(ここまで)
    \end{varwidth}
  };
\end{tikzpicture}
\end{verbatim}
\end{mybox}

\begin{tikzpicture}[overlay,remember picture]
\draw ([shift={(0mm,0mm)}]current page.north east) node[fill=white,draw=black,below left]{\begin{varwidth}{10cm}\footnotesize
\begin{tabular}{rl}
$\mathbf{u}$ : & 単位相対丸め\\
$\mathrm{fl}(\cdot)$ : & 数値計算結果\\
\end{tabular}
\end{varwidth}};
\end{tikzpicture}
\end{frame}

\begin{frame}[containsverbatim]{tikzpictureの応用2}
\begin{itemize}
\item itemizeに装飾をつけるときにも使用可能
\end{itemize}


\begin{mybox}{}\tiny
\begin{verbatim}
% tikz用のマーキングをするコマンド
\newcommand{\MARK}[1]{\tikz[remember picture]\node[inner sep=0pt](#1){\vphantom{X}};\!}
  
\begin{itemize}
  \item \MARK{a}こんな % aでマーキング
  \item 感じ
  \item \MARK{b}です  % bでマーキング
\end{itemize}

\begin{tikzpicture}[overlay,remember picture,very thick]          % 以下のshiftはx,y方向のシフト
\draw[decoration={brace},decorate]                                    % 「--」をbrace({)でデコ
  ([shift={(-4.5mm,0mm)}]b.south) -- ([shift={(-4.5mm,0mm)}]a.north)  % bの南からaの北まで
  coordinate[midway,left=3mm](c)                                      % 中点の左3mmをcとする
  coordinate[midway,left=1.5mm](d);                                   % 中点の左1.5mmをdとする
\draw[->] (d) -- (c) |- ([shift={(0mm,-5mm)}]b.south) node[right]{これらは...}; % 矢印とその右に文字
\end{tikzpicture}
\end{verbatim}
\end{mybox}

\vspace{-2mm}
\newcommand{\MARK}[1]{\tikz[remember picture]\node[inner sep=0pt](#1){\vphantom{X}};\!}
\begin{itemize}
  \item \MARK{a}こんな % aでマーキング
  \item 感じ
  \item \MARK{b}です  % bでマーキング
\end{itemize}

\begin{tikzpicture}[overlay,remember picture,very thick]
\draw[%
  decoration={brace}, % 線の指定
  decorate,           % 指定した線を引く
] ([shift={(-4.5mm,0mm)}]b.south) -- ([shift={(-4.5mm,0mm)}]a.north) coordinate[midway,left=3mm](c) coordinate[midway,left=1.5mm](d);
\draw[->] (d) -- (c) |- ([shift={(0mm,-5mm)}]b.south) node[right] {これらは...};
\end{tikzpicture}
\end{frame}


\backupbegin
\section{補足資料について}
\begin{frame}[containsverbatim]{補足資料}
\begin{itemize}
\item \verb|\backupbegin| $\sim$ \verb|\backupend|で挟んだフレームは総ページ数にカウントしない
\item 質疑応答対策の補足資料などはこの中に書く
\end{itemize}

\begin{mybox}{}\footnotesize
\begin{verbatim}
	\backupbegin          % 総ページ数にカウントしない(ここから)
	\begin{frame}{補足1}
	... 内容 ...
	\end{frame}
	\begin{frame}{補足1}
	... 内容 ...
	\end{frame}
	\begin{frame}{補足1}
	... 内容 ...
	\end{frame}
	\backupend            % 総ページ数にカウントしない(ここまで)
\end{verbatim}
\end{mybox}

\begin{flushright}
ページ番号はこんな感じになる↓
\end{flushright}
\end{frame}
\backupend

\end{document}