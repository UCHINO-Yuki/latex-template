%
% いじらないって書いてあるところ以外はいじくってOK
% by うちの
%

%=====
% 表紙の設定 (任意)
%=====
\newcommand{\TITLE}{たいとる}% タイトル(改行は任意)
\newcommand{\SUBTITLE}{}                           % サブタイトル(ハイフン自動出力)
\newcommand{\YEAR}{2021}                           % 年度
\newcommand{\INSTITUTEa}{理工学研究科}             % 研究科
\newcommand{\INSTITUTEb}{修士課程}                 % 課程
\newcommand{\INSTITUTEc}{システム理工学専攻}       % 専攻
\newcommand{\SUBJECT}{応用数理研究}                % 研究名
\newcommand{\NUMBER}{MF20013}                      % 学番
\newcommand{\FAMILYNAME}{内野}                     % 名字
\newcommand{\FAMILYNAMEruby}{うちの}               % 名字ふりがな
\newcommand{\FIRSTNAME}{佑基}                      % 名前
\newcommand{\FIRSTNAMEruby}{ゆうき}                % 名前ふりがな
\newcommand{\ADVISERFAMILY}{尾崎}                  % 指導教員性
\newcommand{\ADVISERFIRST}{克久}                   % 指導教員名
\newcommand{\POSITION}{教授}                       % 指導教員職位


%=====
% パッケージはここに書く (任意)
%=====
\documentclass[12pt,a4j]{jsarticle}
\usepackage[margin = 20truemm]{geometry}
\usepackage[%                               ハイパーリンク(色なし)
	dvipdfmx,%
	bookmarksopenlevel=4,%
	pdfpagemode=UseThumbs,%
	pdfborder={0 0 0}]{hyperref}
\usepackage{pxjahyper}%                     日本語目次
\usepackage{graphicx}%                      画像入れるなら必須
\usepackage[dvipdfmx]{color}%               画像入れるなら必須
\usepackage{amsthm,amsmath,amssymb,ascmac}% なんかいろいろ
\usepackage{mathtools}%                     数式ツール
\usepackage{bm}%                            数式の太字用
\usepackage{url}%                           参考文献のurl用
\usepackage{here}%                          図表の「H」指定用
\usepackage{multirow}%                      tablar環境用
\usepackage{multicol}%                      tablar環境用
\usepackage{listings}%                      ソースコード用
\usepackage{plistings}%                    日本語ソースコード用
\usepackage[shortlabels]{enumitem}%         カスタマイズ可の箇条書き環境


%=====
% pathの設定 (任意)
%=====
% 「fig」フォルダ内の画像や「program」フォルダ内のコードを参照する場合は
%\graphicspath{{./fig/}}%                    画像のpath
%\lstset{inputpath=program}%                 ソースコードのpath
% を記述する


%=====
% キャプション番号を章番号に依存させる (任意)
% 図1, 図2, 図3, ... => 図1.1, 図1.2, 図2.1, ...
%=====
\renewcommand{\theequation}{\thesection.\arabic{equation}}
\renewcommand{\thefigure}{\thesection.\arabic{figure}}
\renewcommand{\thetable}{\thesection.\arabic{table}}
\makeatletter
\@addtoreset{equation}{section}
\@addtoreset{figure}{section}
\@addtoreset{table}{section}
\AtBeginDocument{%
	\renewcommand*{\thelstlisting}{\arabic{section}.\arabic{lstlisting}}%
	\@addtoreset{lstlisting}{section}%
}%
\makeatother


%=====
% その他の設定 (任意)
%=====
\pagestyle{plain}%                               ページ番号出力
\allowdisplaybreaks%                             数式を自動で改ページする
\setcounter{tocdepth}{3}%                        表示する目次の深さ
\renewcommand{\lstlistingname}{Program}%         ソースコードのキャプションの変更
\renewcommand{\lstlistlistingname}{Program目次}% ソースコード目次の名前
\everymath{\displaystyle}%                      「$~$」をdisplaystyleにする
%\renewcommand{\arraystretch}{1.2}%               表の行間を1.2倍にする
\mathtoolsset{showonlyrefs=true}%                引用した数式にのみ番号を振る
\let\originalleft\left%                          \leftの余白削減
\renewcommand{\left}{\mathopen{}\mathclose\bgroup\originalleft}
\let\originalright\right%                        \rightの余白削減
\renewcommand{\right}{\aftergroup\egroup\originalright}


%=====
% 表紙の出力 (いじらない)
%=====
\newlength{\stwidth}                                  % 長さの変数作成
\settowidth{\stwidth}{\SUBTITLE}                      % \SUBSITLEの長さを取得
\begin{document}
\thispagestyle{empty}                                 % 表紙はページ番号なし
\noindent
\begin{minipage}{\hsize}
\begin{minipage}[c][.7\textheight]{\textwidth}       % 上部70%にタイトル表示
	\centering
	{\LARGE {\YEAR}年度\quad\ 芝浦工業大学大学院}\\[8ex]% 年度と所属
	{\Huge 修\qquad 士\qquad 論\qquad 文}\\[12ex]       % 修士論文
	{\Large 題目:\TITLE}\\[4pt]%                       % タイトル
	\ifnum\stwidth>0%                                   % サブタイトルの長さ>0の場合
	{\Large --- \SUBTITLE ---}                          % サブタイトル出力
	\else%                                              % サブタイトルの長さ=0の場合
	\vspace{1.73zh}%                                    % サブタイトル分の縦スペース
	\fi
\end{minipage}
\begin{minipage}[c][.3\textheight]{\textwidth}       % 下部30%に名前表示
	\centering
	{\large\begingroup
	\renewcommand{\arraystretch}{1.2}%                 % 行間1.2倍
	\begin{tabular}{lp{.5zw}l}
		専\hspace{2zw}攻 && \INSTITUTEa(\INSTITUTEb)\ \INSTITUTEc\\% 専攻
		学籍番号 && \NUMBER\\%                                       % 学番
		ふりがな && \FAMILYNAMEruby\quad \FIRSTNAMEruby\\%           % ふりがな
		氏\hspace{2zw}名 && \FAMILYNAME\quad \FIRSTNAME\\%           % 氏名
		指導教員 && \ADVISERFAMILY\quad \ADVISERFIRST~ \POSITION%    % 指導教員
	\end{tabular}\endgroup}
\end{minipage}
\end{minipage}
\newpage


%=====
% 目次の出力 (任意)
%=====
\pagenumbering{roman}%                 ページ番号をi,ii,iii,...にする
\addcontentsline{toc}{section}{目次}% 「目次」って書く
\tableofcontents%                      目次
\listoffigures%                        図目次
\listoftables%                         表目次
%\lstlistoflistings%                    ソースコード目次
\newpage
\pagenumbering{arabic}%                ページ番号を1,2,3,...にする


%=====
% 以降,本文を書く (任意)
%=====


%=====
% 謝辞 (任意)
%=====
\newpage
\addcontentsline{toc}{section}{謝辞}
\section*{謝辞}
先生ありがとう.みんなありがとう.


%=====
% 参考文献 (bibファイルを使用してもよい)
%=====
\addcontentsline{toc}{section}{参考文献}
\begin{thebibliography}{99}
\bibitem{ref1} 文献1.
\bibitem{ref2} 文献2.
\end{thebibliography}
\end{document}